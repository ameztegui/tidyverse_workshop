\documentclass[]{article}
\usepackage{lmodern}
\usepackage{amssymb,amsmath}
\usepackage{ifxetex,ifluatex}
\usepackage{fixltx2e} % provides \textsubscript
\ifnum 0\ifxetex 1\fi\ifluatex 1\fi=0 % if pdftex
  \usepackage[T1]{fontenc}
  \usepackage[utf8]{inputenc}
\else % if luatex or xelatex
  \ifxetex
    \usepackage{mathspec}
  \else
    \usepackage{fontspec}
  \fi
  \defaultfontfeatures{Ligatures=TeX,Scale=MatchLowercase}
\fi
% use upquote if available, for straight quotes in verbatim environments
\IfFileExists{upquote.sty}{\usepackage{upquote}}{}
% use microtype if available
\IfFileExists{microtype.sty}{%
\usepackage{microtype}
\UseMicrotypeSet[protrusion]{basicmath} % disable protrusion for tt fonts
}{}
\usepackage[margin=1in]{geometry}
\usepackage{hyperref}
\hypersetup{unicode=true,
            pdftitle={Introduction to the tidyverse: dplyr, tidyr and purrr},
            pdfauthor={Aitor Ameztegui, Víctor Granda},
            pdfborder={0 0 0},
            breaklinks=true}
\urlstyle{same}  % don't use monospace font for urls
\usepackage{color}
\usepackage{fancyvrb}
\newcommand{\VerbBar}{|}
\newcommand{\VERB}{\Verb[commandchars=\\\{\}]}
\DefineVerbatimEnvironment{Highlighting}{Verbatim}{commandchars=\\\{\}}
% Add ',fontsize=\small' for more characters per line
\usepackage{framed}
\definecolor{shadecolor}{RGB}{248,248,248}
\newenvironment{Shaded}{\begin{snugshade}}{\end{snugshade}}
\newcommand{\KeywordTok}[1]{\textcolor[rgb]{0.13,0.29,0.53}{\textbf{#1}}}
\newcommand{\DataTypeTok}[1]{\textcolor[rgb]{0.13,0.29,0.53}{#1}}
\newcommand{\DecValTok}[1]{\textcolor[rgb]{0.00,0.00,0.81}{#1}}
\newcommand{\BaseNTok}[1]{\textcolor[rgb]{0.00,0.00,0.81}{#1}}
\newcommand{\FloatTok}[1]{\textcolor[rgb]{0.00,0.00,0.81}{#1}}
\newcommand{\ConstantTok}[1]{\textcolor[rgb]{0.00,0.00,0.00}{#1}}
\newcommand{\CharTok}[1]{\textcolor[rgb]{0.31,0.60,0.02}{#1}}
\newcommand{\SpecialCharTok}[1]{\textcolor[rgb]{0.00,0.00,0.00}{#1}}
\newcommand{\StringTok}[1]{\textcolor[rgb]{0.31,0.60,0.02}{#1}}
\newcommand{\VerbatimStringTok}[1]{\textcolor[rgb]{0.31,0.60,0.02}{#1}}
\newcommand{\SpecialStringTok}[1]{\textcolor[rgb]{0.31,0.60,0.02}{#1}}
\newcommand{\ImportTok}[1]{#1}
\newcommand{\CommentTok}[1]{\textcolor[rgb]{0.56,0.35,0.01}{\textit{#1}}}
\newcommand{\DocumentationTok}[1]{\textcolor[rgb]{0.56,0.35,0.01}{\textbf{\textit{#1}}}}
\newcommand{\AnnotationTok}[1]{\textcolor[rgb]{0.56,0.35,0.01}{\textbf{\textit{#1}}}}
\newcommand{\CommentVarTok}[1]{\textcolor[rgb]{0.56,0.35,0.01}{\textbf{\textit{#1}}}}
\newcommand{\OtherTok}[1]{\textcolor[rgb]{0.56,0.35,0.01}{#1}}
\newcommand{\FunctionTok}[1]{\textcolor[rgb]{0.00,0.00,0.00}{#1}}
\newcommand{\VariableTok}[1]{\textcolor[rgb]{0.00,0.00,0.00}{#1}}
\newcommand{\ControlFlowTok}[1]{\textcolor[rgb]{0.13,0.29,0.53}{\textbf{#1}}}
\newcommand{\OperatorTok}[1]{\textcolor[rgb]{0.81,0.36,0.00}{\textbf{#1}}}
\newcommand{\BuiltInTok}[1]{#1}
\newcommand{\ExtensionTok}[1]{#1}
\newcommand{\PreprocessorTok}[1]{\textcolor[rgb]{0.56,0.35,0.01}{\textit{#1}}}
\newcommand{\AttributeTok}[1]{\textcolor[rgb]{0.77,0.63,0.00}{#1}}
\newcommand{\RegionMarkerTok}[1]{#1}
\newcommand{\InformationTok}[1]{\textcolor[rgb]{0.56,0.35,0.01}{\textbf{\textit{#1}}}}
\newcommand{\WarningTok}[1]{\textcolor[rgb]{0.56,0.35,0.01}{\textbf{\textit{#1}}}}
\newcommand{\AlertTok}[1]{\textcolor[rgb]{0.94,0.16,0.16}{#1}}
\newcommand{\ErrorTok}[1]{\textcolor[rgb]{0.64,0.00,0.00}{\textbf{#1}}}
\newcommand{\NormalTok}[1]{#1}
\usepackage{graphicx,grffile}
\makeatletter
\def\maxwidth{\ifdim\Gin@nat@width>\linewidth\linewidth\else\Gin@nat@width\fi}
\def\maxheight{\ifdim\Gin@nat@height>\textheight\textheight\else\Gin@nat@height\fi}
\makeatother
% Scale images if necessary, so that they will not overflow the page
% margins by default, and it is still possible to overwrite the defaults
% using explicit options in \includegraphics[width, height, ...]{}
\setkeys{Gin}{width=\maxwidth,height=\maxheight,keepaspectratio}
\IfFileExists{parskip.sty}{%
\usepackage{parskip}
}{% else
\setlength{\parindent}{0pt}
\setlength{\parskip}{6pt plus 2pt minus 1pt}
}
\setlength{\emergencystretch}{3em}  % prevent overfull lines
\providecommand{\tightlist}{%
  \setlength{\itemsep}{0pt}\setlength{\parskip}{0pt}}
\setcounter{secnumdepth}{0}
% Redefines (sub)paragraphs to behave more like sections
\ifx\paragraph\undefined\else
\let\oldparagraph\paragraph
\renewcommand{\paragraph}[1]{\oldparagraph{#1}\mbox{}}
\fi
\ifx\subparagraph\undefined\else
\let\oldsubparagraph\subparagraph
\renewcommand{\subparagraph}[1]{\oldsubparagraph{#1}\mbox{}}
\fi

%%% Use protect on footnotes to avoid problems with footnotes in titles
\let\rmarkdownfootnote\footnote%
\def\footnote{\protect\rmarkdownfootnote}

%%% Change title format to be more compact
\usepackage{titling}

% Create subtitle command for use in maketitle
\newcommand{\subtitle}[1]{
  \posttitle{
    \begin{center}\large#1\end{center}
    }
}

\setlength{\droptitle}{-2em}

  \title{Introduction to the tidyverse: \texttt{dplyr}, \texttt{tidyr} and
\texttt{purrr}}
    \pretitle{\vspace{\droptitle}\centering\huge}
  \posttitle{\par}
    \author{Aitor Ameztegui, Víctor Granda}
    \preauthor{\centering\large\emph}
  \postauthor{\par}
      \predate{\centering\large\emph}
  \postdate{\par}
    \date{February 2019}


\begin{document}
\maketitle

\newpage

\section{Introduction}\label{introduction}

Thi is a \texttt{RMarkdown} document generated to illustrate the
exercices suggested as part of the Workshop ``Introduction to the
tidyverse'', that will be celebrated in Barcelona within the 1st Meeting
of the Iberian Ecological Society and the XIV AEET Meeting. All the code
and the data needed to produce this document can be found in GitHub
(\url{https://github.com/ameztegui/dplyr_workshop}). For any doubt about
the exercises, you can contact Aitor Ameztegui
(\url{ameztegui@gmail.com}) or V?ctor Granda
(\url{ameztegui@gmail.com}).

The packages\texttt{tidyr}, \texttt{dplyr}, and \texttt{purrr} are part
of a set of packages known as the
\href{https://cran.r-project.org/web/packages/tidyverse/index.html}{\texttt{tidyverse}},
created by \href{https://github.com/hadley}{Hadley Wickham}, Chief
Scientist in RStudio. The \texttt{tidyverse} was created to easen data
analysis and data management. It consists on packages to import and read
data, to organize and modify them, to analyze and model them and to
visualize them. In this seminar we will focus on \texttt{tidyr},
conceived to help organize data, and \texttt{dplyr}, that focuses on
data transformation. We must firs install these packages (if we haven't
done it yet). To install all the packages from the \texttt{tidyverse} at
once we need to write \texttt{install.packages("tidyverse")}. Then we
need to load the packages by typing:

\begin{Shaded}
\begin{Highlighting}[]
\KeywordTok{library}\NormalTok{(tidyverse)}
\end{Highlighting}
\end{Shaded}

We also need to load the dataset we will use for this workshop, we can
download them from
\href{https://github.com/ameztegui/dplyr_workshop}{GitHub}.

\begin{Shaded}
\begin{Highlighting}[]
\KeywordTok{load}\NormalTok{(}\StringTok{"../data/data_workshop.Rdata"}\NormalTok{)}

\NormalTok{trees <-}\StringTok{ }\KeywordTok{tbl_df}\NormalTok{(trees)}
\NormalTok{plots <-}\StringTok{ }\KeywordTok{tbl_df}\NormalTok{(plots)}
\NormalTok{species <-}\StringTok{ }\KeywordTok{tbl_df}\NormalTok{(species)}
\NormalTok{coordinates <-}\StringTok{ }\KeywordTok{tbl_df}\NormalTok{(coordinates)}
\end{Highlighting}
\end{Shaded}

Through the function \texttt{tbl\_df} we will convert normal data frames
into \emph{tibbles}. A \emph{tibble} is just a data frame with some
particularities: for example, they only print the first 10 rows by
default (instead of the whole data frame), and printing them provides
information on all the variables and their class. Besides that, we can
treat tibbles as normal data frames, because they behave like them at
all effects.

In these exercises we will use four data frames with information from
the 2nd and 3rd Spanish National Forest Inventory (IFN2 e IFN3) in
Catalonia. The data frames are:

\begin{itemize}
\item
  \textbf{plots} {[}11,858 x 15{]}: all the IFN3 plots in Catalonia,
  with info about the date and time of measurement, soil texture and
  soil pH, total canopy cover and tree canopy cover, etc.
\item
  \textbf{trees} {[}111,756 x 12{]}: contains all the adult trees (diam
  \textgreater{} 7.5 cm) measured both in IFN2 and IFN3. Contains info
  about the plot, the species, diameter class, diameter measured at IFN2
  and IFN3\ldots{}
\item
  \textbf{species} {[}14,778 x 15{]}: contains the number of trees/ha
  per species and diameter class.
\item
  \textbf{coordinates} {[}11,858 x 6{]}: contains the X \& Y coordinates
  of each IFN3 plot.
\end{itemize}

First thing to do is to have a look at the data, to get familiar with
the data they contain. We will use the function \texttt{glimpse} for
that.

\begin{Shaded}
\begin{Highlighting}[]
\KeywordTok{glimpse}\NormalTok{(plots)}
\KeywordTok{glimpse}\NormalTok{(trees)}
\KeywordTok{glimpse}\NormalTok{(species)}
\KeywordTok{glimpse}\NormalTok{(coordinates)}
\end{Highlighting}
\end{Shaded}

\newpage

\newpage

\section{\texorpdfstring{\texttt{dplyr}: transforming data
frames}{dplyr: transforming data frames}}\label{dplyr-transforming-data-frames}

\texttt{dplyr} can be used to transform our data frames in the way we
need it in each case: we can create bew variables, select those of
interest, execute filters, etc. The \texttt{dplyr} package contains 5
main verbs:

\begin{itemize}
\item
  \texttt{filter} selects rows based on a given set of conditions
\item
  \texttt{select} select columns based on their name
\item
  \texttt{arrange} sort the data frame based on one or several variables
\item
  \texttt{mutate} create new variables
\item
  \texttt{summarise} create new variables that summarize values of an
  existuing variable (mean, sum, etc.)
\end{itemize}

All of them have a similar structure: the first argument in the function
is the \texttt{data\ frame} to which it will be applied, and the rest of
arguments specify what to do with this \texttt{data\ frame}, depending
on the \texttt{verb}we are using.

\subsection{\texorpdfstring{\texttt{filter}}{filter}}\label{filter}

\texttt{filter} selects those rows of a data frame that accomplish a
certain criterion. The first argument is the data frame, and the rest
are the criteria, that can be specified in chain, separated by commas.

\subsubsection{Exercice 1}\label{exercice-1}

To practice with \texttt{filter} let's try to find those plots of IFN
that:

\begin{itemize}
\tightlist
\item
  1.1 Are located in Barcelona (08) or Girona (17). We have two options:
\end{itemize}

\begin{Shaded}
\begin{Highlighting}[]
\CommentTok{# Option 1}
    \KeywordTok{filter}\NormalTok{ (plots, Provincia }\OperatorTok{==}\StringTok{"08"} \OperatorTok{|}\StringTok{ }\NormalTok{Provincia }\OperatorTok{==}\StringTok{"17"}\NormalTok{)}
\CommentTok{# Option 2}
    \KeywordTok{filter}\NormalTok{ (plots, Provincia }\OperatorTok\StringTok{ }\KeywordTok{c}\NormalTok{(}\StringTok{"08"}\NormalTok{, }\StringTok{"17"}\NormalTok{))}
\end{Highlighting}
\end{Shaded}

We see that both options produce exactly the same result. But the next
option wouldn't be valid, since we need to specify explicitly the
variable every time we add a new condition:

\begin{Shaded}
\begin{Highlighting}[]
\KeywordTok{filter}\NormalTok{(plots, Provincia }\OperatorTok{==}\StringTok{"08"} \OperatorTok{|}\StringTok{ "17"}\NormalTok{)}
\end{Highlighting}
\end{Shaded}

\begin{itemize}
\tightlist
\item
  1.2 plots that were measured completely in January 2001
\end{itemize}

To do this we need to find the plots for which the completion date is
later than 31 December 2000 and earlier than 1 February 2001. We can do
this in two ways: the first one is to use the \texttt{\&} operator to
indicate we want to get the rows that meet both criteria. The second
options would be simply to concatenate both criteria with a comma, since
\texttt{filter} assumes all of them must be met.

\begin{Shaded}
\begin{Highlighting}[]
    \CommentTok{# Option 1}
    \KeywordTok{filter}\NormalTok{ (plots, FechaFin }\OperatorTok{<}\StringTok{ "2001-02-01"} \OperatorTok{&}\StringTok{ }\NormalTok{FechaFin }\OperatorTok{>}\StringTok{ "2000-12-31"}\NormalTok{)}

    \CommentTok{# Option 2}
    \KeywordTok{filter}\NormalTok{ (plots, FechaFin }\OperatorTok{<}\StringTok{ "2001-02-01"}\NormalTok{, FechaFin }\OperatorTok{>}\StringTok{ "2000-12-31"}\NormalTok{)}
\end{Highlighting}
\end{Shaded}

\begin{itemize}
\tightlist
\item
  1.3 Those plots that took more than 2 hours to be measured (7200 seg)
\end{itemize}

\begin{Shaded}
\begin{Highlighting}[]
    \KeywordTok{filter}\NormalTok{(plots, (HoraFin }\OperatorTok{-}\StringTok{ }\NormalTok{HoraIni) }\OperatorTok{>}\DecValTok{7200}\NormalTok{)}
\end{Highlighting}
\end{Shaded}

As we see, we can do operations within \texttt{filter} conditions. In
this case, we want that the difference between EndDate and StartDate be
\textless{} 7200 s (2 hours).

\subsection{\texorpdfstring{\texttt{select}}{select}}\label{select}

\texttt{select} allows to retain only some columns, based on their name.
To help us find the columns, there are some specific functions such as
\texttt{starts\_with} or \texttt{contains}, that only work within
\texttt{select}. We can see the list of special functions by typing
\texttt{help("select")}

\subsubsection{Exercise 2}\label{exercise-2}

To practice with \texttt{select} let's try to find 4 different ways of
selecting the variables that specify the starting and ending date of
measurement of the plots (FechaIni y FechaFin)

\begin{itemize}
\tightlist
\item
  For example, we could specifiy the name of the columns we want to keep
  in an explicit way
\end{itemize}

\begin{Shaded}
\begin{Highlighting}[]
\KeywordTok{select}\NormalTok{(plots, FechaIni, FechaFin)}
\end{Highlighting}
\end{Shaded}

\begin{itemize}
\tightlist
\item
  We can also specify them as a range, so that all columns between the
  two indicated will be selected
\end{itemize}

\begin{Shaded}
\begin{Highlighting}[]
\KeywordTok{select}\NormalTok{(plots, FechaIni}\OperatorTok{:}\NormalTok{FechaFin)}
\end{Highlighting}
\end{Shaded}

\begin{itemize}
\tightlist
\item
  Or we could select all the columns that contain the text `fecha'. In
  this case, since we are not interested in the pH measurement date, we
  can decide to delete it from our selection:
\end{itemize}

\begin{Shaded}
\begin{Highlighting}[]
\KeywordTok{select}\NormalTok{(plots, }\KeywordTok{contains}\NormalTok{ (}\StringTok{"Fecha"}\NormalTok{),}\OperatorTok{-}\NormalTok{FechaPh)}
\end{Highlighting}
\end{Shaded}

\begin{itemize}
\tightlist
\item
  At last, we could also select all the variables that start with
  `fecha' (in this case we also need to eliminate FechaPh):
\end{itemize}

\begin{Shaded}
\begin{Highlighting}[]
\KeywordTok{select}\NormalTok{(plots, }\KeywordTok{starts_with}\NormalTok{(}\StringTok{"Fecha"}\NormalTok{), }\OperatorTok{-}\NormalTok{FechaPh)}
\end{Highlighting}
\end{Shaded}

\subsection{\texorpdfstring{\texttt{arrange}}{arrange}}\label{arrange}

\texttt{arrange} sorts the data frame based on the values of ine or more
variables (columns). The first argument will be, as usual, the data
frame we want to sort, and then we must specify the variables that
determine the ordering. If we specify more than 1 variable, the
succesive variables will be used to decide order when there are ties
(i.e.~secondary sorting variables). We can also use `desc(x)' to sort in
decreasing order. Let's try with a few exercises:

\subsubsection{Exercise 3}\label{exercise-3}

\begin{itemize}
\tightlist
\item
  Ex.3.1 Sort the plots by measurement date and time
\end{itemize}

\begin{Shaded}
\begin{Highlighting}[]
\KeywordTok{arrange}\NormalTok{(plots, FechaFin, HoraFin)}
\end{Highlighting}
\end{Shaded}

\begin{itemize}
\tightlist
\item
  Ex. 3.2 Which plots were started to be measured later in the day?
\end{itemize}

\begin{Shaded}
\begin{Highlighting}[]
\KeywordTok{arrange}\NormalTok{(plots,}\KeywordTok{desc}\NormalTok{(HoraIni))}
\end{Highlighting}
\end{Shaded}

\begin{itemize}
\tightlist
\item
  Ex. 3.3 Which took longer to be measured?
\end{itemize}

\begin{Shaded}
\begin{Highlighting}[]
\KeywordTok{arrange}\NormalTok{(plots, }\KeywordTok{desc}\NormalTok{(HoraFin}\OperatorTok{-}\NormalTok{HoraIni))}
\end{Highlighting}
\end{Shaded}

We see that, as it happens with \texttt{filter}, we can also sort data
frames based on the result of an arithmetic operation.

\subsection{\texorpdfstring{\texttt{mutate}}{mutate}}\label{mutate}

\texttt{mutate} allows us to create new variables with a certain value
or as combination of existing variables. We just need to specify the
data frame, and indicate the new variables name and its value. Let's see
some examples:

\subsubsection{Exercise 4}\label{exercise-4}

Let's create two new variables:

\begin{itemize}
\tightlist
\item
  Ex.4.1 A variable with individual tree growth (in cm) between IFN2 and
  IFN3.
\end{itemize}

\begin{Shaded}
\begin{Highlighting}[]
\NormalTok{trees <-}\StringTok{ }\KeywordTok{mutate}\NormalTok{ (trees, }\DataTypeTok{growth=}\NormalTok{ DiamIf3 }\OperatorTok{-}\StringTok{ }\NormalTok{DiamIf2)}
\end{Highlighting}
\end{Shaded}

\begin{itemize}
\tightlist
\item
  Ex.4.2 Create two new variables with the basal area per hectare that
  each tree represents, both in IFN2 and IFN3. Which species was the
  fastest growing tree in basal area?
\end{itemize}

\begin{Shaded}
\begin{Highlighting}[]
\NormalTok{trees <-}\StringTok{ }\KeywordTok{mutate}\NormalTok{(trees, }\DataTypeTok{BAIf2=}\NormalTok{ (((pi}\OperatorTok{/}\DecValTok{4}\NormalTok{)}\OperatorTok{*}\NormalTok{(DiamIf2}\OperatorTok{/}\DecValTok{100}\NormalTok{)}\OperatorTok{^}\DecValTok{2}\NormalTok{)}\OperatorTok{*}\NormalTok{N),}
                \DataTypeTok{BAIf3=}\NormalTok{ (((pi}\OperatorTok{/}\DecValTok{4}\NormalTok{)}\OperatorTok{*}\NormalTok{(DiamIf3}\OperatorTok{/}\DecValTok{100}\NormalTok{)}\OperatorTok{^}\DecValTok{2}\NormalTok{)}\OperatorTok{*}\NormalTok{N),}
                \DataTypeTok{BA_growth =}\NormalTok{ BAIf3 }\OperatorTok{-}\StringTok{ }\NormalTok{BAIf2)}

\KeywordTok{arrange}\NormalTok{(trees, }\KeywordTok{desc}\NormalTok{(BA_growth))}
\end{Highlighting}
\end{Shaded}

As we see, we can calculate new variables based on the variables we just
created. Also, we can combine \texttt{mutate} and \texttt{arrange} to
know which is the fastest growing tree.

\subsection{\texorpdfstring{\texttt{summarise}}{summarise}}\label{summarise}

\texttt{summarise} allows us to make calculations with the variables in
the data frame, but using \emph{summary functions}, that transform the
variability in a given variable into a single value. Functions such as
\texttt{sum}, \texttt{mean}, \texttt{max},\texttt{IQR}, etc. are
examples of summary functions. However, this function by itself often
lacks any interest, cause it would reduce all the data frame to a single
value. It is commonly used together with \texttt{group\_by}, that
classifies the data frame in groups based on a categorical variable.

To use \texttt{group\_by} we just need to indicate the data frame and
the variable we want to group it by. To be more efficient,
\texttt{dplyr} does not create a copy of the data frame, but it creates
a hidden variable that indexes the groups, so that when we ask it to
perform operations by group, it know to which group belongs each
observation.

In the case of our data frame \texttt{trees}, there are several groups
that could be of interest el caso de nuestra base de datos de pies
trees, hay varios grupos que pueden tener inter?s:

\begin{Shaded}
\begin{Highlighting}[]
\CommentTok{# Por provincia}
\NormalTok{by_province <-}\StringTok{ }\KeywordTok{group_by}\NormalTok{ (trees, Provincia)}

\CommentTok{# Por parcela}
\NormalTok{by_plot <-}\StringTok{ }\KeywordTok{group_by}\NormalTok{ (trees, Codi)}

\CommentTok{# Por especie}
\NormalTok{by_species <-}\StringTok{ }\KeywordTok{group_by}\NormalTok{ (trees, Especie)}

\CommentTok{#Por clase diam?trica}
\NormalTok{by_CD <-}\StringTok{ }\KeywordTok{group_by}\NormalTok{ (trees, CD)}

\CommentTok{#Por parcela y especie}
\NormalTok{by_plot_species <-}\StringTok{ }\KeywordTok{group_by}\NormalTok{ (trees, Codi, Especie)}
\end{Highlighting}
\end{Shaded}

We can see, by typing \texttt{glimpse(by\_plot)} that the resulting data
frame is not different at all from the original, at least apparently.
However, if we type \texttt{class(by\_plot)} we see it has now a new
class \texttt{grouped\_df}.

\subsubsection{Exercise 5}\label{exercise-5}

What statistics could be of interest to characterize the diameter values
of each plot? We can, for example, calculate te mean, minimum and
maximum value, percentile 0.9 and interquartile range for each plot. We
can also compute the number of trees measured in each plot and the
number of different species, using the functions \texttt{n()} y
\texttt{n\_distinct(x)}. In this case, the resulting data frame will
have less rows, one per plot, and will only contain the new variables
created.

\begin{Shaded}
\begin{Highlighting}[]
\KeywordTok{summarise}\NormalTok{(by_plot,}
          \DataTypeTok{media =} \KeywordTok{mean}\NormalTok{(DiamIf3),}
          \DataTypeTok{min =} \KeywordTok{min}\NormalTok{ (DiamIf3),}
          \DataTypeTok{max =} \KeywordTok{max}\NormalTok{(DiamIf3),}
          \DataTypeTok{q90 =} \KeywordTok{quantile}\NormalTok{(DiamIf3, }\FloatTok{0.9}\NormalTok{),}
          \DataTypeTok{IQ =} \KeywordTok{IQR}\NormalTok{(DiamIf3),}
          \DataTypeTok{n =}\KeywordTok{n}\NormalTok{(),}
          \DataTypeTok{sps =} \KeywordTok{n_distinct}\NormalTok{(Especie) )}
\end{Highlighting}
\end{Shaded}

\begin{verbatim}
## # A tibble: 7,713 x 8
##    Codi   media   min   max   q90    IQ     n   sps
##    <fct>  <dbl> <dbl> <dbl> <dbl> <dbl> <int> <int>
##  1 080001  26.3  13.4  38    34.4  4.15    15     1
##  2 080002  35.2  24.8  44.4  43.0 10.2     13     3
##  3 080003  32.0  14.2  51    46.1 12.2      7     2
##  4 080004  24.3  16.8  31.7  30.2  7.42     2     1
##  5 080005  28.4  16.2  59.8  39.7 15.3     12     3
##  6 080006  35.9  14    55.9  52.3 17.9     23     2
##  7 080007  30.8  15.2  63.6  49.4 12.3     35     2
##  8 080008  16.0   9    21.4  17.8  2.05    11     1
##  9 080009  16.7   9    36.0  24    3.88    16     2
## 10 080010  31.6   9.2  95.5  61.5 18.2     13     3
## # ... with 7,703 more rows
\end{verbatim}

\newpage

\section{\texorpdfstring{Pipelines
(\texttt{\%\textgreater{}\%})}{Pipelines (\%\textgreater{}\%)}}\label{pipelines}

We will often use several \texttt{dplyr} verbs together, creating nested
functions. However, when we need to perform several operations, these
nested functions can easily get complex and difficult to understand. For
example, by having a look at this code, would you be able to say what it
will do?

\begin{Shaded}
\begin{Highlighting}[]
\NormalTok{diam_medio_especie <-}\StringTok{ }\KeywordTok{filter}\NormalTok{(}
    \KeywordTok{summarise}\NormalTok{(}
        \KeywordTok{group_by}\NormalTok{(}
            \KeywordTok{filter}\NormalTok{(}
\NormalTok{                trees,}
                \OperatorTok{!}\KeywordTok{is.na}\NormalTok{(DiamIf3)}
\NormalTok{            ),}
\NormalTok{        Codi, Especie}
\NormalTok{        ),}
    \DataTypeTok{diam =} \KeywordTok{mean}\NormalTok{ (DiamIf3),}
    \DataTypeTok{n =} \KeywordTok{n}\NormalTok{()}
\NormalTok{    ),}
\NormalTok{n }\OperatorTok{>}\StringTok{ }\DecValTok{5}\NormalTok{)}
\end{Highlighting}
\end{Shaded}

The code gets those observations, from the data frame \texttt{trees},
that have a value of diameter (\texttt{!is.na(DiamIf3)}), it then groups
them by plot and species (\texttt{group\_by(Codi,\ Especie)}),
calculates for each combination the mean diameter
(\texttt{diam\ =\ mean\ (DiamIf3)}), and the number of trees per plot
(\texttt{n\ =\ n()}), and finally selects only those cases in which
there are at least 5 trees (\texttt{filter\ (n\textgreater{}5)}).

Although this syntaxis is not operationally complex, it is hard to
understand. Often a solution is to save each step as a different data
frame, but this is an important source of errors.

We can however simplify this code using the \emph{pipe} operator
(\texttt{\%\textgreater{}\%}) from the \texttt{magrittr} package, which
is installed and loaded with \texttt{tidyr} and \texttt{dplyr}. When we
use \texttt{\%\textgreater{}\%}, the result of the left side is
processed by the righ side function as first argument. In the case of
\texttt{dplyr} and \texttt{tidyr}, since the first argument is always a
data frame, \texttt{\%\textgreater{}\%} makes that a function be applied
to the data frame resulting from the previous function. Thus, we can
express \texttt{filter\ (df,\ color\ =="blue)} as
\texttt{df\ \%\textgreater{}\%\ filter(color\ ==\ "blue")}. This allows
to concatenate several functions in a logical and understandable way, so
that the operator \texttt{\%\textgreater{}\%} could be read as
\emph{then}. Let's see how this would be in the previous function

\begin{Shaded}
\begin{Highlighting}[]
\NormalTok{diam_medio_especie <-}\StringTok{ }\NormalTok{trees }\OperatorTok\StringTok{             }\CommentTok{# take the df 'trees' and THEN}
\KeywordTok{filter}\NormalTok{(}\OperatorTok{!}\KeywordTok{is.na}\NormalTok{(DiamIf3)) }\OperatorTok\StringTok{                 }\CommentTok{# eliminate NA values and THEN}
\KeywordTok{group_by}\NormalTok{(Codi, Especie) }\OperatorTok\StringTok{                 }\CommentTok{# group y plot and species and THEN}
\KeywordTok{summarise}\NormalTok{(}\DataTypeTok{diam=}\KeywordTok{mean}\NormalTok{(DiamIf3), }\DataTypeTok{n =} \KeywordTok{n}\NormalTok{()) }\OperatorTok\StringTok{  }\CommentTok{# calculate mean and number of trees and THEN}
\KeywordTok{filter}\NormalTok{(n }\OperatorTok{>}\StringTok{ }\DecValTok{5}\NormalTok{)                               }\CommentTok{# filter those with n> 5}
\end{Highlighting}
\end{Shaded}

\subsubsection{Exercise 6}\label{exercise-6}

Let's do some exercises. Using the pipe operator, let's create pipelines
to solve the next exercises:

\begin{itemize}
\tightlist
\item
  Ex.6.1 Which plots have the greatest average growth between IFN2 and
  IFN3?
\end{itemize}

We first define the data frame we will work with. \emph{THEN}
(\texttt{\%\textgreater{}\%}) we create a new variable with the growth
of each tree, \emph{THEN} we group by plot, \emph{THEN} we calculate,
for each plot, the mean growth, and \emph{THEN} we arrange the results
in decreasing order. The resulting code would be:

\begin{Shaded}
\begin{Highlighting}[]
\NormalTok{trees }\OperatorTok
\StringTok{    }\KeywordTok{mutate}\NormalTok{(}\DataTypeTok{growth=}\NormalTok{DiamIf3}\OperatorTok{-}\NormalTok{DiamIf2) }\OperatorTok
\StringTok{    }\KeywordTok{group_by}\NormalTok{(Codi) }\OperatorTok
\StringTok{    }\KeywordTok{summarise}\NormalTok{(}\DataTypeTok{av_growth=}\KeywordTok{mean}\NormalTok{(growth), }\DataTypeTok{n=}\KeywordTok{n}\NormalTok{()) }\OperatorTok
\StringTok{    }\KeywordTok{arrange}\NormalTok{(}\KeywordTok{desc}\NormalTok{(av_growth))}
\end{Highlighting}
\end{Shaded}

\begin{verbatim}
## # A tibble: 7,713 x 3
##    Codi   av_growth     n
##    <fct>      <dbl> <int>
##  1 171089      23.1     3
##  2 170819      21.6     1
##  3 172607      17.6     6
##  4 172216      17.4     6
##  5 172690      16.0    17
##  6 171682      15.4     6
##  7 083267      15.3     1
##  8 431363      15.1     4
##  9 171664      14.8     5
## 10 171976      14.4     1
## # ... with 7,703 more rows
\end{verbatim}

\begin{itemize}
\tightlist
\item
  Ex.6.2 Which is the plot with highest species richness?
\end{itemize}

First, we define thedata frame (\texttt{trees}), \emph{THEN} we group by
Code, \emph{THEN} we determine the number of species per plot and
\emph{THEN} we arrange in decreasing order:

\begin{Shaded}
\begin{Highlighting}[]
\NormalTok{trees }\OperatorTok
\StringTok{    }\KeywordTok{group_by}\NormalTok{(Codi) }\OperatorTok
\StringTok{    }\KeywordTok{summarise}\NormalTok{ (}\DataTypeTok{n_species=}\KeywordTok{n_distinct}\NormalTok{(Especie)) }\OperatorTok
\StringTok{    }\KeywordTok{arrange}\NormalTok{(}\KeywordTok{desc}\NormalTok{(n_species))}
\end{Highlighting}
\end{Shaded}

\begin{verbatim}
## # A tibble: 7,713 x 2
##    Codi   n_species
##    <fct>      <int>
##  1 170195         9
##  2 171036         9
##  3 170218         8
##  4 170121         7
##  5 170596         7
##  6 170635         7
##  7 170799         7
##  8 171398         7
##  9 171481         7
## 10 172650         7
## # ... with 7,703 more rows
\end{verbatim}

\begin{itemize}
\tightlist
\item
  Ex.6.3 Are both variables (species richness and average growth)
  related?
\end{itemize}

First, we would need to indicate the data frame we will work with,
\emph{THEN} we will group by plot, \emph{THEN} we will calculate the
variables. To see the relationship between both variables, we will
create a plot with \texttt{ggplot2}, just to show how all the packages
in the tidyverse relate to each other. The aim of this workshop is not
to learn \texttt{ggplot2}, so we won't go into further details. To know
more about ggplot you can visit this website: \url{http://ggplot2.org/}.

\begin{Shaded}
\begin{Highlighting}[]
\NormalTok{trees }\OperatorTok
\StringTok{    }\KeywordTok{mutate}\NormalTok{(}\DataTypeTok{growth=}\NormalTok{DiamIf3}\OperatorTok{-}\NormalTok{DiamIf2) }\OperatorTok
\StringTok{    }\KeywordTok{group_by}\NormalTok{(Codi) }\OperatorTok
\StringTok{        }\KeywordTok{summarise}\NormalTok{ (}\DataTypeTok{n_species=}\KeywordTok{n_distinct}\NormalTok{(Especie),}
                   \DataTypeTok{av_growth=}\KeywordTok{mean}\NormalTok{(growth)) }\OperatorTok
\StringTok{    }\KeywordTok{ggplot}\NormalTok{(}\KeywordTok{aes}\NormalTok{(n_species, av_growth)) }\OperatorTok{+}
\StringTok{    }\KeywordTok{geom_point}\NormalTok{() }\OperatorTok{+}
\StringTok{    }\KeywordTok{geom_smooth}\NormalTok{(}\DataTypeTok{method =} \StringTok{"lm"}\NormalTok{)}
\end{Highlighting}
\end{Shaded}

\includegraphics{workshop_exercices_files/figure-latex/unnamed-chunk-20-1.pdf}

Here we see one of the advantages of the \texttt{tidyverse}, the fact
that all the packages and functions can communicate one with another. In
this way, we just created a plot without the need of creating
intermediate objects or data frames, starting directly from the raw data
frame, and chianing orders in a logical and intuitive way.

\section{Grouped mutate/grouped
filter}\label{grouped-mutategrouped-filter}

Most of the times we use \texttt{group\_by}, we will do it with the
\emph{summary functions}, that is, functions that take n values as
input, and give back 1 value as output. Examples of \emph{summary
functions} are \texttt{mean()}, \texttt{sd()}, \texttt{min()},
\texttt{sum()}, etc.

However, some times we will need to do some operation by group, but we
will need to produce one output per input, that is n inputs
--\textgreater{} n outputs. This can be done using \texttt{mutate} or
\texttt{filter} in combination with \texttt{group\_by}.

\subsubsection{Exercise 7}\label{exercise-7}

Taking this into account, let's try to:

\begin{itemize}
\tightlist
\item
  Ex.7.1 Identify those trees that grow much faster than the average of
  the plot
\end{itemize}

\begin{Shaded}
\begin{Highlighting}[]
\NormalTok{trees }\OperatorTok
\StringTok{    }\KeywordTok{mutate}\NormalTok{(}\DataTypeTok{growth=}\NormalTok{DiamIf3}\OperatorTok{-}\NormalTok{DiamIf2) }\OperatorTok
\StringTok{    }\KeywordTok{group_by}\NormalTok{(Codi) }\OperatorTok
\StringTok{    }\KeywordTok{mutate}\NormalTok{(}\DataTypeTok{plot_mean=} \KeywordTok{mean}\NormalTok{(growth),}
           \DataTypeTok{des =}\NormalTok{ (growth }\OperatorTok{-}\StringTok{ }\NormalTok{plot_mean)) }\OperatorTok
\StringTok{    }\KeywordTok{arrange}\NormalTok{(}\KeywordTok{desc}\NormalTok{(des))}
\end{Highlighting}
\end{Shaded}

In the previous code we see we first calculate the growth of each tree,
and after grouping by plot, we calculate a new variable, where the plot
average is substracted from the growth of each tree, and the result is
divided by the standard deviation of the plot. We calculate in this way
the standardized growth of each tree with respect to the plot, making it
easy to identify those trees that grow suspiciously more than the
average for their plot.

\begin{itemize}
\tightlist
\item
  Ex. 7.2 Identify those plots where a species grows much more than the
  average for the species
\end{itemize}

\begin{Shaded}
\begin{Highlighting}[]
\NormalTok{trees }\OperatorTok\StringTok{ }
\StringTok{    }\KeywordTok{mutate}\NormalTok{(}\DataTypeTok{growth=}\NormalTok{DiamIf3}\OperatorTok{-}\NormalTok{DiamIf2) }\OperatorTok
\StringTok{    }\KeywordTok{group_by}\NormalTok{(Especie) }\OperatorTok
\StringTok{    }\KeywordTok{mutate}\NormalTok{(}\DataTypeTok{growth_sp =} \KeywordTok{mean}\NormalTok{(growth)) }\OperatorTok
\StringTok{    }\KeywordTok{group_by}\NormalTok{(Codi, Especie) }\OperatorTok
\StringTok{    }\KeywordTok{mutate}\NormalTok{(}\DataTypeTok{growth_sp_plot =} \KeywordTok{mean}\NormalTok{(growth),}
           \DataTypeTok{inc =}\NormalTok{ (growth_sp_plot }\OperatorTok{/}\NormalTok{growth_sp))}\OperatorTok
\StringTok{    }\KeywordTok{arrange}\NormalTok{(}\KeywordTok{desc}\NormalTok{(inc))}
\end{Highlighting}
\end{Shaded}

As we did before, we first calculate the growth of eaxh tree, we then
group by species, so that we can calculate the mean growth for each
species (\texttt{growth\_sp}). Finally, we group again, now for plot and
species, to calculate the mean growth of each species on each plot
(\texttt{growth\_sp\_plot}). Once we have this, we can calculate the
ratio between the two variables, identifying those plots where the
species is performing better )assuming no mistakes, of course).

Let's see one last example:

*Ex.7.3 Select those species of the IFN3 occupied by ``pure'' Pinus
nigra stands (Especie = 025)

Note: a forest is considered as pure stand if more than 80\% of their
Basal Area corresponds to a single species. Let's see how we would do
that:

\begin{Shaded}
\begin{Highlighting}[]
\NormalTok{trees }\OperatorTok
\StringTok{    }\KeywordTok{group_by}\NormalTok{(Codi,Especie) }\OperatorTok
\StringTok{    }\KeywordTok{summarise}\NormalTok{(}\DataTypeTok{BA_sp=} \KeywordTok{sum}\NormalTok{(BAIf3)) }\OperatorTok
\StringTok{    }\KeywordTok{group_by}\NormalTok{(Codi) }\OperatorTok
\StringTok{    }\KeywordTok{mutate}\NormalTok{(}\DataTypeTok{BA_tot =} \KeywordTok{sum}\NormalTok{(BA_sp),}
           \DataTypeTok{ratio=}\NormalTok{ BA_sp}\OperatorTok{/}\NormalTok{BA_tot) }\OperatorTok
\StringTok{    }\KeywordTok{filter}\NormalTok{(Especie}\OperatorTok{==}\StringTok{"025"}\NormalTok{, ratio }\OperatorTok{>}\FloatTok{0.8}\NormalTok{)}
\end{Highlighting}
\end{Shaded}

\begin{verbatim}
## # A tibble: 648 x 5
## # Groups:   Codi [648]
##    Codi   Especie BA_sp BA_tot ratio
##    <fct>  <fct>   <dbl>  <dbl> <dbl>
##  1 080132 025     41.4   49.9  0.830
##  2 080307 025     51.2   54.8  0.933
##  3 080313 025     28.1   35.0  0.804
##  4 080318 025     23.7   26.6  0.894
##  5 080322 025     29.7   32.3  0.919
##  6 080323 025      1.61   1.61 1    
##  7 080324 025     18.2   19.7  0.924
##  8 080325 025     43.8   44.6  0.982
##  9 080326 025      6.01   6.01 1    
## 10 080328 025     62.1   65.7  0.945
## # ... with 638 more rows
\end{verbatim}

In this case, we first calculate BA per plot, using \texttt{summarise}.
We then calculate the sum of BA per plot, but in this case we use
\texttt{mutate}, because we don't want to aggregate tge data by plot,
but calculate them separately for each plot but keeping the rest of the
data as it was. Once we have both values, we can filter to select those
plots with Pinus nigra, in which percenrage of basal area for that
species be \textgreater{} 80\%. \newpage

\newpage

\section{Joins: working with to
tables}\label{joins-working-with-to-tables}

Very often, the information we will work with more than a table. The
\emph{join} functions will allow us to work with several data frames,
joining them in different ways. Within \texttt{dplyr} there are two
types of joins:

\subsection{Mutating joins}\label{mutating-joins}

They add the columns of a data frame to the other, depending on whether
they share some observations or not. There are four types.

\begin{itemize}
\item
  \texttt{left\_join(x,\ y)} adds the columns of \texttt{y} to the
  observations of \texttt{x} that are also in \texttt{y}. Those that are
  not present in y will receive the value \texttt{NA}. With this
  function we ensure that we will not lose any observation.
\item
  \texttt{right\_join(x,\ y)} adds the columns of \texttt{x} to those
  observations in \texttt{y} that are also in \texttt{x}. Those that are
  not present will receive \texttt{NA}. It is equivalent to
  \texttt{left\_join}, but the columns will be ordered differently.
\item
  \texttt{full\_join(x,y)} includes all observations in \texttt{x} and
  \texttt{y}. If they do not coincide, they assign \texttt{NA}.
\item
  \texttt{inner\_join(x,\ y)} includes only those observations both in
  \texttt{x} and \texttt{y} (repeats rows if it is necessary).
\end{itemize}

\subsection{Filtering joins}\label{filtering-joins}

The second type of joins are the \textbf{filtering joins}, that affect
only to the observation, not to the variables. That is, they never add
new columns, but they keep or delete the rows of the original frame as a
function of their correspondence or not with a second data frame. There
are only two types:

\begin{itemize}
\tightlist
\item
  \texttt{semi\_join(x,\ y)} \emph{keeps} the observations in \texttt{x}
  that match observations in \texttt{y}.
\item
  \texttt{anti\_join(x,\ y)} \emph{deletes} the observations in
  \texttt{x} that match observations in \texttt{y}.
\end{itemize}

You can find more information about the join functions typing
\texttt{vignette("two-table")}.

\subsubsection{Exercise 8}\label{exercise-8}

To try the join functions, let's add the geographic information (X \& Y
coordinates), contained in the data frame \texttt{coordinates} to the
data frame \texttt{plots}.

\begin{Shaded}
\begin{Highlighting}[]
\KeywordTok{left_join}\NormalTok{(plots,coordinates, }\StringTok{"Codi"}\NormalTok{)}
\end{Highlighting}
\end{Shaded}

In this case, since we want to keep all the plots in the original
\texttt{plots}data frame, we use \texttt{left\_join}. In this case,
since the number of observations in \texttt{coordinates} and
\texttt{plots} is the same, the function \texttt{inner\_join} should
give us the same results.

Now we added the coordinates, we can represent in a map any variable in
the data frame. We could, for example, represent the values of tree
canopy cover (FccArb). We need to load the package ``maps''. (If we
don't have it installed, we can install it typing in the console
\texttt{install.packages("maps")}).

\begin{Shaded}
\begin{Highlighting}[]
\KeywordTok{library}\NormalTok{(maps)}
\end{Highlighting}
\end{Shaded}

\begin{verbatim}
## 
## Attaching package: 'maps'
\end{verbatim}

\begin{verbatim}
## The following object is masked from 'package:purrr':
## 
##     map
\end{verbatim}

\begin{Shaded}
\begin{Highlighting}[]
\KeywordTok{left_join}\NormalTok{(plots,coordinates, }\StringTok{"Codi"}\NormalTok{) }\OperatorTok
\KeywordTok{ggplot}\NormalTok{( }\KeywordTok{aes}\NormalTok{(CoorX, CoorY)) }\OperatorTok{+}
\StringTok{    }\KeywordTok{geom_point}\NormalTok{(}\KeywordTok{aes}\NormalTok{(}\DataTypeTok{color=}\NormalTok{FccArb), }\DataTypeTok{size=}\FloatTok{0.3}\NormalTok{, }\DataTypeTok{alpha=}\FloatTok{0.6}\NormalTok{) }\OperatorTok{+}
\StringTok{    }\KeywordTok{scale_color_continuous}\NormalTok{(}\DataTypeTok{low=} \StringTok{"white"}\NormalTok{, }\DataTypeTok{high=}\StringTok{"dark green"}\NormalTok{)}
\end{Highlighting}
\end{Shaded}

\includegraphics{workshop_exercices_files/figure-latex/Ex.8b: joins (cont)-1.pdf}

Once again, we see we don't even need to create a new data frame with
the new information, we can chain the functions in \texttt{dplyr} and
\texttt{ggplot2}, producing the results in a very easy and fast way.

\newpage

\section{\texorpdfstring{\texttt{tidyr}: changing the shape of the data
frames}{tidyr: changing the shape of the data frames}}\label{tidyr-changing-the-shape-of-the-data-frames}

One of the main ideas behind the
\href{https://cran.r-project.org/web/packages/tidyverse/index.html}{\texttt{tidyverse}}
is the concept of \emph{tidy data}, that we have introduced before.
According to Hadley Wickham, we can say that our data are tidy when two
conditions are met:

\begin{itemize}
\item
  Each column corresponds to a variable
\item
  Each row is a different observation
\end{itemize}

Of course, the data are not always organized in this way, sometimes
other formats are more efficient (for example, for gathering data). For
instance, if we have a look at the table \texttt{species}we will see
that the number of trees for the different size classes are in different
columns. This format is more convenient for entering the data or for
some kinds of analyses, but in general the \emph{tidy} format easens the
processing and analysis, specially in vectorized languages such as
\texttt{R}.

\begin{Shaded}
\begin{Highlighting}[]
\CommentTok{# View(species)}
\end{Highlighting}
\end{Shaded}

The \texttt{tidyr} package allows to change the way in which data are
organized, so that we can arrange them in the way we need to our
analysis. It has four basic verbs:

\begin{itemize}
\item
  \texttt{gather} aggregates variables that are in several columns and
  converts them into two variables: a factor (\emph{key}) and a numeric
  variablea (\emph{value}).
\item
  \texttt{spread} is the inverse of gather, it takes the levels of a
  factor and a numeric variable and creates a new variable for each
  level of the factor.
\item
  \texttt{separate} divides the content of a column into several columns
\item
  \texttt{unite} inverse of separate, concatenates the values of several
  columns
\end{itemize}

\subsection{\texorpdfstring{\texttt{gather} \&
\texttt{separate}}{gather \& separate}}\label{gather-separate}

\texttt{gather} transforms data in \emph{wide} format into \emph{long}
format. \texttt{gather} takes a series of columns and transforms them
into two variables: a factor (\emph{key}) and a numeric variable
(\emph{value}). The first parameter in \texttt{gather} is the data
frame, the second and third are the names we will give to \emph{key} and
\emph{value}, and the rest are the variables to group.

\subsubsection{Exercise 9}\label{exercise-9}

Let's use \texttt{gather} and \texttt{separate} to transform the data
frame \texttt{species} into a \emph{tidy} format, where each column is a
variable and each row, an observation. First, let's have a look at the
data frame we want to transform:

\begin{Shaded}
\begin{Highlighting}[]
\KeywordTok{glimpse}\NormalTok{(species)}
\end{Highlighting}
\end{Shaded}

To convert it into the `long' format we specify the data frame, then the
new factor to create (\emph{key}), and the numeric variable
(\emph{value}), and last, the columns to aggregate. For the last part we
have three equivalent options:

\begin{itemize}
\item
  \begin{enumerate}
  \def\labelenumi{(\Alph{enumi})}
  \tightlist
  \item
    Explicitly define the variables we want to gather:
  \end{enumerate}
\end{itemize}

\begin{Shaded}
\begin{Highlighting}[]
    \KeywordTok{gather}\NormalTok{(species, CD, n, CD_}\DecValTok{10}\NormalTok{,CD_}\DecValTok{15}\NormalTok{,CD_}\DecValTok{20}\NormalTok{, CD_}\DecValTok{25}\NormalTok{,CD_}\DecValTok{30}\NormalTok{,}
\NormalTok{                      CD_}\DecValTok{35}\NormalTok{,CD_}\DecValTok{40}\NormalTok{, CD_}\DecValTok{45}\NormalTok{,CD_}\DecValTok{50}\NormalTok{,CD_}\DecValTok{55}\NormalTok{,CD_}\DecValTok{60}\NormalTok{, CD_}\DecValTok{65}\NormalTok{, CD_}\DecValTok{70}\NormalTok{)}
\end{Highlighting}
\end{Shaded}

\begin{itemize}
\item
  \begin{enumerate}
  \def\labelenumi{(\Alph{enumi})}
  \setcounter{enumi}{1}
  \tightlist
  \item
    Define the interval containing the variables we want to gather
  \end{enumerate}
\end{itemize}

\begin{Shaded}
\begin{Highlighting}[]
    \KeywordTok{gather}\NormalTok{(species, CD, n, CD_}\DecValTok{10}\OperatorTok{:}\NormalTok{CD_}\DecValTok{70}\NormalTok{)}
\end{Highlighting}
\end{Shaded}

\begin{itemize}
\item
  \begin{enumerate}
  \def\labelenumi{(\Alph{enumi})}
  \setcounter{enumi}{2}
  \tightlist
  \item
    Define the variables with the helpers functions
  \end{enumerate}
\end{itemize}

\begin{Shaded}
\begin{Highlighting}[]
    \KeywordTok{gather}\NormalTok{(species, CD, n, }\KeywordTok{starts_with}\NormalTok{(}\StringTok{'CD'}\NormalTok{))}
\end{Highlighting}
\end{Shaded}

\begin{itemize}
\item
  \begin{enumerate}
  \def\labelenumi{(\Alph{enumi})}
  \setcounter{enumi}{3}
  \tightlist
  \item
    Define the variables we DO NOT want to include (with -). The
    function will assume we want to gather the rest of variables.
  \end{enumerate}
\end{itemize}

\begin{Shaded}
\begin{Highlighting}[]
     \KeywordTok{gather}\NormalTok{(species,CD, n,}\OperatorTok{-}\NormalTok{Codi, }\OperatorTok{-}\NormalTok{Especie)}
\end{Highlighting}
\end{Shaded}

The three pieces of code above produce the same result. Once we have
converted the data frame into the new format, we can divide the new
variable ``CD'' into two new variables, that we will name ``Name'' and
``CD'', using \texttt{separate}. If we do not specify where to make the
separation, the function takes by defualt the first non alphanumeric
character in the string.

\begin{Shaded}
\begin{Highlighting}[]
\NormalTok{    species_long <-}\StringTok{   }\KeywordTok{gather}\NormalTok{(species,CD, n,}\OperatorTok{-}\NormalTok{Codi, }\OperatorTok{-}\NormalTok{Especie)}
\NormalTok{    species_long<-}\KeywordTok{separate}\NormalTok{(species_long, }\DataTypeTok{col=}\NormalTok{CD, }\DataTypeTok{into =} \KeywordTok{c}\NormalTok{(}\StringTok{"Nombre"}\NormalTok{, }\StringTok{"CD"}\NormalTok{))}
\NormalTok{    species_long}
\end{Highlighting}
\end{Shaded}

\subsection{\texorpdfstring{\texttt{spread} \&
\texttt{unite}}{spread \& unite}}\label{spread-unite}

If we have a data frame in \emph{long} format, we can use
\texttt{spread} and \texttt{unite} to transform it back into the
\emph{wide} format. This is what we will do in the next exercise,
converting back the data frame with species and diameter classes into
its original format. As with \texttt{gather}, \texttt{spread} takes the
data frame as the first argument. The secod parameter is the factor we
will use to create the new columns, and the third parameter is the name
of the column that contains the values. We can see this with an example:

\subsubsection{Exercise 10}\label{exercise-10}

Use \texttt{unite} and \texttt{spread} to transform back the data to its
origianl format.

First we create a new variable, that will be useful to create the new
columns:

\begin{Shaded}
\begin{Highlighting}[]
\NormalTok{species_unite <-}\StringTok{ }\KeywordTok{unite}\NormalTok{(species_long, CD, Nombre, CD)}
\end{Highlighting}
\end{Shaded}

Now we will transform the data frame, specifying the variable that will
produce the new columns (``CD'') and the variable that contains the
values (n)

\begin{Shaded}
\begin{Highlighting}[]
\KeywordTok{spread}\NormalTok{(species_unite, CD,n)}
\end{Highlighting}
\end{Shaded}

\section{Other interesting functionalities of the
tidyverse}\label{other-interesting-functionalities-of-the-tidyverse}

\subsection{Communication between
packages}\label{communication-between-packages}

All the packages in the tidyverse are designed to communicate with each
other. That means that we can combine \texttt{dplyr} and \texttt{tidyr}
functions, connecting them trhough pipes (\texttt{\%\textgreater{}\%}).
For example, if we wanted to know which the mean diameter distribution
of all the pine species we could first filter the species we want to
study (pines) then transform the data frame, group the data by species
and diameter class and finally calculate the mean number of trees:

\begin{Shaded}
\begin{Highlighting}[]
\NormalTok{species }\OperatorTok
\StringTok{    }\KeywordTok{filter}\NormalTok{(Especie }\OperatorTok\StringTok{ }\KeywordTok{c}\NormalTok{(}\StringTok{"021"}\NormalTok{, }\StringTok{"022"}\NormalTok{,}\StringTok{"023"}\NormalTok{, }\StringTok{"024"}\NormalTok{, }\StringTok{"025"}\NormalTok{, }\StringTok{"026"}\NormalTok{)) }\OperatorTok
\StringTok{    }\KeywordTok{gather}\NormalTok{(CD, n, CD_}\DecValTok{10}\OperatorTok{:}\NormalTok{CD_}\DecValTok{70}\NormalTok{) }\OperatorTok
\StringTok{    }\KeywordTok{group_by}\NormalTok{(Especie,CD) }\OperatorTok
\StringTok{    }\KeywordTok{summarise}\NormalTok{(}\DataTypeTok{n_trees=}\KeywordTok{mean}\NormalTok{(n))}
\end{Highlighting}
\end{Shaded}

\begin{verbatim}
## # A tibble: 78 x 3
## # Groups:   Especie [?]
##    Especie CD    n_trees
##    <fct>   <chr>   <dbl>
##  1 021     CD_10   55.4 
##  2 021     CD_15  121.  
##  3 021     CD_20  138.  
##  4 021     CD_25  109.  
##  5 021     CD_30   63.7 
##  6 021     CD_35   29.8 
##  7 021     CD_40   13.2 
##  8 021     CD_45    6.43
##  9 021     CD_50    3.00
## 10 021     CD_55    1.39
## # ... with 68 more rows
\end{verbatim}

But \texttt{dplyr} and \texttt{tidyr} also can connect to other packages
in the \texttt{tidyverse}, such as \texttt{ggplot2} or \texttt{broom},
so we could expand on the previous code to generate a plot by species.

\begin{Shaded}
\begin{Highlighting}[]
\NormalTok{species }\OperatorTok
\StringTok{    }\KeywordTok{filter}\NormalTok{(Especie }\OperatorTok\StringTok{ }\KeywordTok{c}\NormalTok{(}\StringTok{"021"}\NormalTok{, }\StringTok{"022"}\NormalTok{,}\StringTok{"023"}\NormalTok{, }\StringTok{"024"}\NormalTok{, }\StringTok{"025"}\NormalTok{, }\StringTok{"026"}\NormalTok{)) }\OperatorTok
\StringTok{    }\KeywordTok{gather}\NormalTok{(CD, n, CD_}\DecValTok{10}\OperatorTok{:}\NormalTok{CD_}\DecValTok{70}\NormalTok{) }\OperatorTok
\StringTok{    }\KeywordTok{group_by}\NormalTok{(Especie,CD) }\OperatorTok
\StringTok{    }\KeywordTok{summarise}\NormalTok{(}\DataTypeTok{n_trees=}\KeywordTok{mean}\NormalTok{(n)) }\OperatorTok
\StringTok{    }\KeywordTok{ggplot}\NormalTok{(}\KeywordTok{aes}\NormalTok{(}\DataTypeTok{x=}\NormalTok{CD, }\DataTypeTok{y=}\NormalTok{n_trees)) }\OperatorTok{+}
\StringTok{    }\KeywordTok{geom_col}\NormalTok{() }\OperatorTok{+}
\StringTok{    }\KeywordTok{facet_wrap}\NormalTok{(}\OperatorTok{~}\NormalTok{Especie)}
\end{Highlighting}
\end{Shaded}

\includegraphics{workshop_exercices_files/figure-latex/ejemplo understanding 2-1.pdf}

\subsection{Functional sequences}\label{functional-sequences}

Another interesting aspect of \texttt{dplyr}is that we can save
sequences of orders as an object, so they can later be applied to
different data frames, as if it was a function. To do this, we must use
the pronoun \texttt{.} as data frame in the sequence of orders to save.
Let's see an example:

\begin{Shaded}
\begin{Highlighting}[]
\NormalTok{         av_growth <-}\StringTok{  }\NormalTok{. }\OperatorTok
\StringTok{            }\KeywordTok{mutate}\NormalTok{(}\DataTypeTok{growth=}\NormalTok{DiamIf3}\OperatorTok{-}\NormalTok{DiamIf2) }\OperatorTok
\StringTok{            }\KeywordTok{group_by}\NormalTok{(Codi) }\OperatorTok
\StringTok{            }\KeywordTok{summarise}\NormalTok{(}\DataTypeTok{mean=}\KeywordTok{mean}\NormalTok{(growth), }\DataTypeTok{n=}\KeywordTok{n}\NormalTok{())}
\end{Highlighting}
\end{Shaded}

If we print the object, we will see it has a class
\texttt{functional\ sequence}, and it specifies the orders to execute:

\begin{Shaded}
\begin{Highlighting}[]
\NormalTok{av_growth}
\end{Highlighting}
\end{Shaded}

\begin{verbatim}
## Functional sequence with the following components:
## 
##  1. mutate(., growth = DiamIf3 - DiamIf2)
##  2. group_by(., Codi)
##  3. summarise(., mean = mean(growth), n = n())
## 
## Use 'functions' to extract the individual functions.
\end{verbatim}

We can then apply this sequence to a data frame\ldots{}

\begin{Shaded}
\begin{Highlighting}[]
\NormalTok{    trees }\OperatorTok\StringTok{ }\KeywordTok{av_growth}\NormalTok{()}
\end{Highlighting}
\end{Shaded}

\begin{verbatim}
## # A tibble: 7,713 x 3
##    Codi    mean     n
##    <fct>  <dbl> <int>
##  1 080001  3.33    15
##  2 080002  3.63    13
##  3 080003  5.93     7
##  4 080004  6.55     2
##  5 080005  2.08    12
##  6 080006  2.28    23
##  7 080007  2.45    35
##  8 080008  1.79    11
##  9 080009  1.86    16
## 10 080010  3.33    13
## # ... with 7,703 more rows
\end{verbatim}

\ldots{} or combine it with new \texttt{dplyr} or \texttt{tidyr}
functions

\begin{Shaded}
\begin{Highlighting}[]
\NormalTok{    trees }\OperatorTok
\StringTok{        }\KeywordTok{filter}\NormalTok{(Provincia}\OperatorTok{==}\StringTok{"17"}\NormalTok{) }\OperatorTok
\StringTok{        }\KeywordTok{av_growth}\NormalTok{()}
\end{Highlighting}
\end{Shaded}

\begin{verbatim}
## # A tibble: 2,113 x 3
##    Codi    mean     n
##    <fct>  <dbl> <int>
##  1 170004  2.99    39
##  2 170005  1.86    28
##  3 170006  1.75    31
##  4 170007  3.14    29
##  5 170008  1.68     9
##  6 170009  1.48    28
##  7 170010  2.40    26
##  8 170012  2.11    21
##  9 170013  1.55    34
## 10 170014  5.38     8
## # ... with 2,103 more rows
\end{verbatim}

\subsection{Databases}\label{databases}

In this workshop we've seen how to work with \texttt{dplyr} and
\texttt{tidyr} using data stored in our computer, but \texttt{dplyr}also
allows working woth remote databases, admiting most formats and
standards: PostgreSQL, MySQL, SQLite, MonetDB, BigQuery, Oracle\ldots{}

When working with databases we will use the same verbs and coding we've
seen so far, but \texttt{dplyr}transforms the \texttt{R} code into
\texttt{SQL}sequences, so we don't need to change the language to read
and analyze the data. Also, it is much faster than \texttt{R}. The
details of working with databases are beyond the scope of this workshop,
but you can find more information in the resources listed in the
following section.

\newpage

\section{More info}\label{more-info}

Both the code and the data needed to generate this document and execute
the examples can be found in GitHub
(\url{https://github.com/ameztegui/dplyr_workshop}). You can also find
more information about these packages and their functions in the book
\emph{\href{http://r4ds.had.co.nz/}{R for data science}} by Hadley
Wickham, or in the vignettes for each function.

\begin{Shaded}
\begin{Highlighting}[]
\CommentTok{# Sobre dplyr}
\KeywordTok{vignette}\NormalTok{(}\StringTok{"introduction"}\NormalTok{)}

\CommentTok{# Sobre tidyr}
\KeywordTok{vignette}\NormalTok{(}\StringTok{"tidy-data"}\NormalTok{)}

\CommentTok{# Sobre unir dos tablas mediante join}
\KeywordTok{vignette}\NormalTok{(}\StringTok{"two-table"}\NormalTok{)}

\CommentTok{# Sobre trabajo con databases}
\KeywordTok{vignette}\NormalTok{(}\StringTok{"databases"}\NormalTok{)}
\end{Highlighting}
\end{Shaded}


\end{document}
